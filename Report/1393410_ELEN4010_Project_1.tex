\documentclass[12pt,twocolumn]{IEEEtran}
\usepackage{ifpdf}
\usepackage{graphicx}
\usepackage{cite}

\ifpdf
\pdfinfo{
	/Title (<<>>)
	/Author (Nicholas Kastanos (1393410))
	/CreationDate (D:201902211949)
	/ModDate (D:201902211949)
	/Subject (Software 3 Report)
	/Keywords (<<>>)
}
\fi

\begin{document}
	\onecolumn
	\title{Analysis of Dependency Injection Frameworks and their Features}
	
	\author{\IEEEauthorblockN{Nicholas Kastanos (1393410)}\\
		\IEEEauthorblockA{School of Electrical \& Information Engineering, University of the Witwatersrand, Private Bag 3, 2050, Johannesburg, South Africa\\ELEN4010 Software Development III\\<<>>}}
	
	\maketitle
	
	\begin{abstract}
		
	\end{abstract}

\thispagestyle{empty}
\setcounter{page}{0}
\twocolumn
\clearpage

\section{Introduction}

\section{Dependency Injection}

The origins of Dependency Injection are rooted in the concepts of the \textit{Dependency Inversion Principle} (DIP) which was first discussed by Martin in 1996 \cite{martin1996}, and the \textit{Inversion of Control} (IoC) principle, by Foote \cite{foote1988}. 

%"High-level modules should not depend upon low-level modules. Both should depend upon abstractions" R. C. Martin. The Dependency Inversion Principle. C++ Report, 8(6):61–66, June 1996.

% IoC --------------- R. E. Johnson and B. Foote. Designing reusable classes. Journal of Object-Oriented Programming, 1(2):22–35, 1988.

These principles state that classes should not depend on other concrete classes, but should only depend on abstract interfaces. This allows the programmer to avoid what Martin describes as a ``Bad Design'', and avoids rigidity, frailty,and immobility in the code.

By having classes depend on abstract interfaces, the implementation of the dependencies can be easily altered without needing the change the code of the class. This makes the code more flexible and less fragile, leading the easier future developments or changes. 

Dependency Injection (DI) provides the solution to this problem by linking the required implementation to the interface without the dependant class having to have knowledge of which implementation is being used \cite{tonymarston2016,hongyulyangewantemperohaydenmelton2008}. 

% MAYBE TALK ABOUT ORDERS OF LAYERS AND DEPENDENCIES AND HOW DI MAKES THIS BETTER.
It is possible to implement DI using the standard libraries for many languages, however this requires well structured and planned code, and can become complicated and prone to errors in large programs. In order to simplify the inclusion of DI in code, many DI Frameworks have been developed.

\section{DI Frameworks}

% Talk about frameworks in general, and then DI frameworks. How do they help implement DI and IoC effectively.
DI frameworks are structured to encourage good use of the underlying principles \cite{rodjohnson2005,dennisdoomen2018}, and are designed to make the use of DI principles easy and simple. DI Frameworks in Java are usually implemented in two structures: XML configuration files, or Injector Java objects. Some of the more popular DI Frameworks are discussed below.
% Many Dependency Injection Frameworks are designed to make the implementation of IoC and DIP principles into a unified and simple

Spring Framework is a comprehensive framework which provides more functionality than DI. It also supports features such as data binding, database connections, and web framework support. The DI Framework section of Spring makes use of both Plain Old Java Objects (POJO) anbd XML files to configure the bindings and injections.
% https://docs.spring.io/spring-framework/docs/current/spring-framework-reference/core.html#beans-factory-collaborators

Guice is a lightweight DI Framework developed by Google to provide fast and simple DI using POJO principles. Unlike Spring Framework, Guice only provides DI capabilities.

Dagger 2 is a DI-only framework which is fully static and is implemented at compile time. This fully implements the program's stack at compile time which allows for easier debugging and faster run-times. Additionally it as able to run on Android systems, which increases the reach of the framework.
% https://google.github.io/dagger/

\section{Features of DI Frameworks} \label{sec:features}
Many DI frameworks have both similar and different features or types of DI which it can implement. These features can either be classified as Essential Features or Supplementary Features.

Essential Features are central to implementing the DIP and IoC principles, and are generally included in every DI Framework. Supplementary Features, however, provide additional utility or Quality of Life improvements to assist in implementing more complicated DI structures. 

% https://www.vogella.com/tutorials/SpringDependencyInjection/article.html

\subsubsection{Constructor Injection} % Guice, Spring, Dagger

Constructor Injection occurs when the object is injected into the constructor as an argument. This allows an object to be provided with the required dependency on construction without violating the DI Principle. Spring Framework, Guice and Dagger 2 all offer this form of injection. 

\subsubsection{Method Injection} % Guice, Spring, Dagger 

Similarly to Constructor Injection, Method Injection occurs when the dependency is passed into a method as an argument. This form of injection differs from Constructor Injection by creating the dependency when the method is called, as opposed to when the object is constructed. This interaction may result in more distributed computational requirements should the construction be computationally expensive. Spring Framework, Guice and Dagger 2 all offer this form of injection. 

\subsubsection{Field Injection} % Guice, Dagger

Field Injection allows a dependency to be injected directly into an object's fields. This generally occurs on instantiation of the object, and allows the constructor of the object to be free of some injections. Guice and Dagger 2 contain this functionality.

\subsubsection{Defaults or Optional Injection} % Spring, Guice, Dagger
Default values allow dependency injection to become optional, where the dependency does not have to be created. If the injection does not occur, the program will be able to use a default value for the required dependency. This feature encourages reuse of code, and reduces fatal program errors, however it may increase the amount of logical errors which are present. All three of the frameworks mentioned above have this feature.

\subsubsection{Named Linkings} % Guice, Spring, Dagger
This feature allows links between objects and their dependencies to be referred to by more 'human readable' names. This increases the readability of code, and does not afford extra functionality. While this does not provide the frameworks with more capabilities, Spring, Guice, and Dagger 2 all support this feature.

\subsubsection{Chained Dependencies} % Spring
Chained dependencies occur when one object's dependency contains objects which itself is dependent on. While this is possible to implement using Guice and Dagger 2, Spring Framework provides active support for is feature.

\subsubsection{Provider Injections} % All
Provider injections can be used when multiple values, or objects, need to be provided to an object. This ensures than a different instance of the dependency is provided each time. Spring Framework, Guice and Dagger 2 all offer this form of injection. 

\subsection{Essential Features}
According to the guidelines set out in Section~\ref{sec:features}, Constructor Injection, Method Injection, and Provider Injection are Essential Features for a DI Framework.

Each of these methods of injection provide a framework for implementing the DIP and IoC Principles. Constructor Injection allows dependencies to be set up at the beginning of the object's lifetime, while Method Injection allows the dependencies to be created in the middle of the lifetime. 

Provider Injections are also essential due to the ability to inject large amounts of data and dependencies. Without it, objects would be limited in the amount of dependencies which can be created. 

\subsection{Supplementary Features}

The remaining features, Field and Optional Injection, Named Linkings and Chained Dependencies, can be classified as Supplementary Features. Field Injection can be implemented using Constructor or Method injection by passing a dependency into the objects as a parameter, and setting it to a private member. Optional Injections removes the need for a default injected object, which reduces the complexity of the code. Chained Dependencies can be complicated to program and test without the support of the framework. This may result in longer development times or lower product quality.

\section{Practical Implementation using Guice}

In order to explore the features and usability of DI Frameworks, a sample Java program has been developed using Google's Guice framework. % INSERT GIT REPO HERE
The sample program emulates how different payment modules can be integrated into a payment or credit transaction system. The user is presented with a login screen, where a username and password is accepted. Once the user has logged in, the user is placed into a loop where they can repeatedly make transactions until the logout. 

The type of transaction is implemented using the Guice DI Framework. An administrator account can change the \verb|Transaction| class's implementation between a \verb|SecureTransaction|, \verb|FastTransaction|, and \verb|InsecureTransaction|. This implementation is then injected into the \verb|PaymentService|'s constructor via a Provider class, and is used to complete the transaction using different methods. 

\section{Analysis of DI Frameworks}


\section{Conclusion}



\bibliographystyle{IEEEtran}
\bibliography{IEEEabrv,references}


\clearpage
\onecolumn
\appendices
\renewcommand\thefigure{\thesection.\arabic{figure}} 
\renewcommand\theequation{\thesection.\arabic{equation}} 
\renewcommand\thetable{\thesection.\arabic{table}}
\section{Program Screenshots}
\setcounter{figure}{0}  
\setcounter{equation}{0} 
\setcounter{table}{0}


\clearpage
\section{Source Code}
\setcounter{figure}{0}  
\setcounter{equation}{0} 
\setcounter{table}{0}

	
\end{document}